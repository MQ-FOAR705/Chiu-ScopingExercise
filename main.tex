

\documentclass{article}

\title{Proof of Concept Scoping Exercise}
\author{Osmond Chiu}


\begin{document}
\maketitle
\section*{Planned Thesis Topic}

My planned thesis research will be inductive research that aims to understand re-nationalisations in Australia. Re-nationalisations are when privatised government functions are brought back under public ownership.\par
The aim is to work out why these re-nationalisations occurred through compiling a list of re-nationalisations to identify common factors and examining key case studies.

\section*{Jobs}

There are a range of jobs that will need to be done for my thesis. The jobs will include (but not be limited to):
\begin{itemize}
\item Writing my research proposal;
\item Developing a research plan;
\item Collecting existing literature;
\item Doing a literature review;
\item Compiling a list of examples of re-nationalisation in Australia;
\item Identifying key Australian case studies;
\item Collecting background information on case studies;
\item Coding data on re-nationalisations;
\item Analysing the data;
\item Writing my first draft;
\item Editing the draft;
\item Proofing; and
\item Referencing and doing the bibliography.
\end{itemize}

\section*{Pains}

There are a range of potential pain points during this research process based on the jobs that must be done. \par
\subsection*{Research plan}
The research plan will inevitably change as previously planned research tasks may not be possible or there may be delays. For example, it may be hard to identify individuals involved in the re-nationalisation case studies and contact them to be interviewed. Alternatively, the research question may need to change if there is not enough information or if there is too much. This may lead to the revision of research plans or even the research question based on available evidence, which might throw out existing timelines and increase time pressures.\par
As I am doing the MRes part-time while also working, pain points for my work may also affect my thesis research plan. Some work pain points are predictable but others are not as my work can be reactive which might create its own thesis pain points.
\subsection*{Data Collection}
The data collection process will be time consuming. To my knowledge, there is no comprehensive, existing list or databases of re-nationalisations in Australia. A list will need to constructed from scratch. It will require trawling through a range of sources including books, journal articles, newspaper articles and government documents but there still may not be enough relevant information for Australian case studies.\par
Difficulties and challenges will include making decisions about what to explore further based on available information. There are risks such missing key information by not searching or wasting time by going down a rabbit hole that does not address my research question, adding stress.\par 
\subsection*{Literature review}
Writing the literature review will involve a lot of work summarising and synthesising the data collected and require tapping into knowledge of all the data and background reading that was previously done. Keeping track of what was written in the key sources without re-reading will be a challenge.
\subsection*{Coding data}
Any quantitative data analysis may rely on me first coding information from qualitative sources. This will be a time consuming task that may take longer than expected as I will need to determine what the relevant and common factors are that should be analysed. There may be definition issues and I will need to spend a lot of time producing metadata to explain what I did. 
\subsection*{Analysing data}
My data analysis will rely heavily on qualitative data. While sources can be identified, decisions about privatisation and ownership are highly political and influenced by the broader context. Understanding why decisions are made, for example, by applying theoretical frameworks such as Marxism or rational choice theory, will require reading theoretical texts and making conclusions based on available evidence which is subjective.\par
\subsection*{Writing}
Writing the thesis will be a time consuming process in itself. Staying disciplined and writing a large body of work to multiple deadlines will be a test. It may not be straight forward as I am likely to make many changes to the structure of the thesis and the content based on feedback from my supervisor. Any changes that are lost between versions will add stress and use up finite time.\par
\subsection*{Editing and Proofing}
I will need to involve others and get assistance for feedback, editing and proofing because it will be difficult to do myself. Relying on others will mean I need to fit into their schedules which may clash with my existing timelines.\par
\subsection*{Formatting}
Formatting the thesis could be a painful process if formatting is changed after the thesis is completed. It could be extremely time consuming and cause problems if reformatting is required to make sure the submitted thesis complies with requirements.\par
\subsection*{Referencing}
Managing referencing, citations and doing a bibliography will be challenging. It is easy to make a mistake such as leaving out a citation or using an incorrect referencing style when dealing with hundreds of references.\par 

\section*{Pain relievers}

There are some opportunities to ease these pain points.\par
\subsection*{Research plan}
Clear and achievable objectives and timelines are needed to avoid unrealistic planning. Keeping track on what tasks need to be done and when by may assist. Being clear about dependencies, priorities and lists could help determine what tasks need to be done and in which order. This would help to allocate appropriate time for research, editing and correcting, balanced against other non-thesis commitments such as work.
\subsection*{Data collection}
Having a clear deadline of when data collection needs to be done by might help combat perfectionism and rein in research by forcing an end date to get the thesis completed on time. 
\subsection*{Literature review}
Keeping track of the sources I have read, what its main arguments were and my criticisms of them will be important. Anything that can assist me will be a big help in writing the literature review so I do not have to re-read sources. 
\subsection*{Writing}
During the writing process, help to more efficiently use time and avoid wasting would be great. Managing version control for my thesis will also be handy to ensure that nothing is completely lost when versions are edited after comments from my supervisor.\par
\subsection*{Formatting}
Being able to easily change the format of my thesis easily to meet journal or other requirements without causing formatting problems would ease stresses and save time during edits.
\subsection*{Referencing}
Reducing the amount of work needed to correctly reference in the correct style and automatically generating a bibliography would free up additional time to work on other tasks and ease stress.\par

\section*{Gains}

There are some opportunities for gains to be made.
\subsection*{Research Plan}

It would be a big help to have some assistance to manage my time and tasks for my thesis and other competing non-thesis commitments.\par

\subsection*{Data Collection}

Manually searching for keywords and potentially related sources is extremely time consuming and I may miss relevant sources.  It would be great if I could automate searches to identify sources I might miss in Trove, Hansard, websites, government documents, academic journals and Google Books. It would be a big help and improve the quality of my research.\par

\subsection*{Literature review}

It would be helpful to have some assistance in keeping track of sources for my literature review and organising the different ideas and arguments from those sources.\par

\subsection*{Formatting}

Not having to worry about correct formatting because correct templates for my thesis are available would ease concerns.\par

\subsection*{Referencing}

It would be great if I could easily change my referencing styles and the bibliography was automatically generated so I did not have to manually keep track of citations.\par

\section*{Gain creators}
Some of the gains previously listed could be delivered on with the right tools.
\subsection*{Research Plan}
It would be a big help to have some useful tools for planning that help me to develop a thesis timeline and enable me to balance my thesis and non-thesis work and personal commitments. Ideally it would be tools that could easily take into account changes to research plans and give time estimates. \par

\subsection*{Data Collection}

A tool that would help me automate searching for keywords in books, journals, newspapers and websites would improve the quality of my research and also eliminate a tedious task that would take significant time. It would make it easier to keep the metadata and identify what keywords I did not search for.

\subsection*{Literature review}
A tool that could assist in both tracking and giving a good structure to concepts from sources for my literature review would be helpful.

\subsection*{Formatting}
A tool that provides a template with the correct format for my thesis would be great.

\subsection*{Referencing}

A tool that automatically swaps the format of my references and my bibliography would be good. It could also make it easier to submit my thesis or a section of it to journals in the future that use different referencing styles.

\section*{Out of Scope}

A number of these jobs that cause pain are likely to be out of scope for this project because they require human judgement, are subjective and may not be easy to break up into definable step-by-step tasks. For example, making decisions that influence data collection such as the scope of my research question or definitions of re-nationalisation, analysing summarised sources for literature reviews, trying to apply broader theoretical frameworks and understand of the broader context to case studies and involving others in proofing and editing.\par
Of the jobs that could relieve pain or generate gains that could be broken down and have a tool or technique applied such as assistance with a research plan, better data organisation for a literature review or making formatting and referencing easier, automating data collection would benefit me the most. It would result in substantially improved research for my thesis and also free up time from one of the more arduous as well as tedious job. I have also chosen it because it is more difficult for me to work out how to go about doing this job and assistance would be helpful. 

\section*{Computational analysis}
It is important to breaking down this data collection job into discrete tasks and steps. 

\subsection*{Decomposing}
There are a range of discrete tasks for this job. It will involve breaking down into:
\begin{itemize}
    \item defining what data needs to be collected
    \item determining the available resources to conduct the research
    \item conducting the research
    \item collecting the data
    \item analysing the collected data
    \item determining if the process needs to be repeated
\end{itemize}

\subsection*{Pattern Recognition}
The main pattern would be the repetitive nature of these tasks because it would need to occur multiple times to obtain the data necessary using different parameters. 

\subsection*{Algorithm}
Step-by-step, this job would likely involve:
\begin{enumerate}
\item Defining the research aim
\item Determining parameters such as time period, location and keywords for research
\item Identifying available sources and databases
\item Researching what tools interact with those sources and databases
\item Testing the tools
\item Using the tool, if it works
\item Saving sources
\item Producing metadata
\item Organising sources
\item Analysing sources
\item Deciding if the process needs to be repeated because the research aim is not met
\end{enumerate}

\end{document}