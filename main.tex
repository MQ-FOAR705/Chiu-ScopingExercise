

\documentclass[a4paper]{article}

\title{Proof of Concept Scoping Exercise}
\author{Osmond Chiu}


\begin{document}
\maketitle
\section*{Planned Thesis Topic}

My planned thesis research will be inductive research that aims to understand re-nationalisations in Australia. Re-nationalisations are when privatised government functions are brought back under public ownership.\par
The aim is to work out why these re-nationalisations occurred through compiling a list of re-nationalisations to identify common factors and examining key case studies. The research may also extend to whether Australia is different to other comparable countries in having more or fewer re-nationalisations and the reasons why this might be so.

\section*{Tasks to be done for thesis}

There are a range of tasks that will need to be done for my thesis. The tasks will include (but not be limited to):
\begin{itemize}
\item Developing a research question and plan;
\item Writing my research proposal;
\item Collecting data;
\item Doing a literature review;
\item Analysing the data;
\item Writing a dissertation plan
\item Writing my first draft;
\item Editing the draft;
\item Proofing; and
\item Referencing and doing a bibliography.
\end{itemize}
Collecting data and doing the literature review will involve (but not be limited to):
\begin{itemize}
    \item Compiling a list of examples of re-nationalisation in Australia and finding relevant further background information;
    \item Identifying key Australian case studies to explore in further detail including the broader socio-economic context;
    \item Collating existing literature on re-nationalisations in Australian and overseas; and
    \item Developing questions and conducting interviews with key individuals and organisations involved in Australian case studies.
\end{itemize}
Analysing data may involve (but not be limited to):
\begin{itemize}
    \item Coding data on key factors applicable to re-nationalisations to identify any patterns;
	\item Applying theoretical frameworks to determine if they provide good explanations for re-nationalisations; and
\item Examining interview transcripts to identify any common themes mentioned by interviewees.
\end{itemize}

\section*{Potential pain points}

There are a range of potential pain points during this research process based on the tasks that must be done. \par
\subsection*{Data Collection}
The data collection process for the literature review will be time consuming. To my knowledge, there is no comprehensive, existing list or databases of re-nationalisations in Australia. In contrast, there is plenty on privatisation including compilations of lists by Parliamentary Libraries. This compares to overseas where organisations such as the Transnational Institute maintain a global database of remunicipalisations or re-nationalisations at a local government level (but this does not include Australia). It will require trawling through a range of sources including books, journal articles, newspaper articles and government documents but there still may not be enough relevant information for Australian case studies.\par
Difficulties and challenges will include making decisions about what to explore further based on available information. There are risks such missing key information by not searching or wasting time by going down a rabbit hole that does not address my research question, adding stress.\par 
Previously planned research tasks may not be possible. It may be hard to identify individuals involved in the re-nationalisation case studies and contact them to be interviewed. This may lead to the revision of research plans based on available evidence, which might throw out existing timelines and increase time pressures.\par
\subsection*{Data Analysis}
My data analysis will rely heavily on qualitative data.  While sources can be identified, decisions about privatisation and ownership are highly political and influenced by the broader context. Understanding why decisions are made, for example, by applying theoretical frameworks such as Marxism or rational choice theory, will require reading theoretical texts and making conclusions based on available evidence which cannot be sped up using technology.\par
Any quantitative data analysis will rely on me first coding information from qualitative sources. This will be a time consuming task that may take longer than expected as I will need to determine what the relevant and common factors are that should be analysed. 
\subsection*{Writing}
Writing the thesis will be a time consuming process in itself. Staying disciplined and writing a large body of work to multiple deadlines will be a test. It may not be straight forward as I am likely to make many changes to the structure of the thesis and the content based on feedback from my supervisor\par
I will need to involve others and get assistance for feedback, editing and proofing because it will be difficult to do myself. Relying on others will mean I need to fit into their schedules which may clash with my existing timelines.\par
Finally, before the thesis is due, ensuring it is in the correct format and referenced properly may also be an extremely time consuming process.\par
\subsection*{Work Commitments}
As I am doing the MRes part-time while also working, pain points for my work may also affect my thesis preparation. Some work pain points are predictable but others are not as my work can be reactive. Managing my time effectively will be important to enable me to focus on one thing at a time or it will create its own thesis pain points.

\section*{Potential pain relievers}

There are some opportunities to ease these pain points by saving time through more efficient methods and planning and scheduling tasks.\par
\subsection*{Managing Tasks and Work Commitments}
Clear and achievable objectives are needed to avoid unrealistic planning. Using project management software such as Jira, Trello or Asana to keep track on what tasks need to be done and when by may assist. Functions such as dependencies, priorities and lists could help me determine what tasks need to be done and in which order. This would help to allocate appropriate time for research, editing and correcting, balanced against other non-thesis commitments such as work.
\subsection*{Writing}
During the writing process, using apps such as Pomodoro to help minimise procrastination will be helpful to more efficiently use time. The use of Github to manage version control for my thesis will also be handy to ensure that nothing is completely lost when versions are edited.\par
\subsection*{Referencing and formatting}
A common thesis mistake is leaving referencing and formatting to the last minute. Reducing the amount of work needed to correctly reference and format could free up additional time to work on other tasks and ease stress. The use of bibliographic referencing software such as EndNote, BibTeX or Zotero may help with referencing. LaTex would be useful for typesetting and help with formatting through the entire process.\par

\section*{Out of scope}

There are a number of tasks which could become pain points that may be out of scope because it will be difficult to use technology to relieve the pain points. This includes applying theoretical frameworks as part of data analysis and making decisions about whether to explore further or limit the scope of my research.

\section*{Opportunities to make gains}

Despite Australia's history as a major privatiser of government assets and services, there is limited existing research on re-nationalisations in Australia. My planned thesis research, if done well, may become a useful resource for trade unions, government, advocacy groups and other academics who may wish to draw on case studies I identify.\par
The use of technology may help improve the quality of research that I conduct by improving my data collection.\par

\par
\subsection*{Data Collection}

The limited existing research means I will need to search for sources that are not in academic journals or books to compile information for my list of re-nationalisations and case studies. Manually searching for keywords and potentially related sources is extremely time consuming and I may miss relevant sources.  Processes and scripts that automate searches could identify sources I might miss in Trove, Hansard, websites, government documents, academic journals and Google Books. It would be a big help and improve the quality of my research.



\end{document}